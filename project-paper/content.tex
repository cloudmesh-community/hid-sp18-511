% status: 0
% chapter: TBD

\title{Automated Spark Cluster Deployment on EC2}


\author{Sandeep Khandelwal}
\affiliation{%
  \institution{Indiana University}
  \city{Bloomington} 
  \state{IN} 
  \postcode{47408}
  \country{USA}}
\email{skhande@iu.edu}


% The default list of authors is too long for headers}
\renewcommand{\shortauthors}{Sandeep}


\begin{abstract}

This project is about developing a module for automated deployment of
Apache Spark cluster on AWS (Amazon Web
Services) EC2 (Elastic Compute
Cloud) instances on a single click and perform various data related tasks on Spark. This
module will take care of different infrastructure related items like
provisioning of EC2 instance, setting up
the public/private key pair for secure connection, inbound/outbound
traffic rules on provisioned machines to allow connectivity on various
ports. This module will also perform Apache Spark installation on the provisioned
EC2 instances using master/worker model. One of the machine will work as a
master node and other machines will work as worker node. The script will expose 
options for Apache Spark cluster deployment, execution of commands on Apache
Spark cluster, and termination of machines.

\end{abstract}

\keywords{AWS, EC2, Apache Spark, Apache Hadoop, Ansible}


\maketitle

\section{Introduction}

AWS~\cite{hid-sp18-511-www-aws} is a cloud service provider that
provides different types of on-demand services to the
customers. AWS~\cite{hid-sp18-511-www-aws} provides various services
to the customers in Infrastructure as a Service, Platform as a Service
and Software as a Service categories. EC2~\cite{hid-sp18-511-www-ec2}
is a compute service provided by AWS~\cite{hid-sp18-511-www-aws} for
easy and fast provisioning of compute resources of different type
based on the requirement. Apache Spark~\cite{hid-sp18-511-www-spark}
is an open source project which use in-memory data for fast and
optimized operation and use Apache
Hadoop~\cite{hid-sp18-511-www-hadoop} as underlying
framework. Ansible~\cite{hid-sp18-511-www-ansible} is an open source
automation platform for configuration management and application
deployment. Ansible~\cite{hid-sp18-511-www-ansible} tool will be used
in this project to deploy Apache Spark~\cite{hid-sp18-511-www-spark}
cluster on EC2~\cite{hid-sp18-511-www-ec2} instances and perform
various tasks on Apache Spark~\cite{hid-sp18-511-www-spark}.

\section{Technology Used}
This section describes the technology used in the project.

\subsection{AWS}

\TODO{Put information about AWS}

\paragraph{EC2}

\TODO{Put information about EC2}

\paragraph{Security Group}

\TODO{Put information about AWS Security Group}

\paragraph{Key Pair}

\TODO{Put information about AWS Key Pair}

\subsection{Apache Spark}

\TODO{Put information about Apache Spark and Master-Slave model}

\subsection{Ansible}

\TODO{Put information about Ansible}

\section{Deploy Apache Spark Cluster}

\TODO{Provide command and screen shots about Apache Spark Cluster deployment}

\section{Execute Command on Apache Spark Cluster}

\TODO{Provide command and screen shots about Command execution on Apache Spark Cluster}

\section{Terminate Apache Spark Cluster}

\TODO{Provide command and screen shots about Apache Spark Cluster termination}

\section{Results}
\TODO{Add results in the table. Specify the duration taken in the deployment and termination of the instances}

\begin{table}[hbt]
	\centering
	\caption{Results}\label{t:results-table}
	\begin{tabular}{llll}
	\end{tabular}
\end{table}


\section{Conclusion}

\TODO{Put here an conclusion. Conclusion and abstracts must not have any
	citations in the section.}


\begin{acks}

  The authors would like to thank Dr.~Gregor~von~Laszewski for his
  support and suggestions to write this paper.

\end{acks}

\bibliographystyle{ACM-Reference-Format}
\bibliography{report} 

