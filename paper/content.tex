\title{Welcome to Informatica Intelligent Cloud Services}

\author{Sandeep Khandelwal}
\affiliation{%
  \institution{Indiana University}
  \city{Bloomington} 
  \state{IN} 
  \postcode{47408}
  \country{USA}
}
\email{skhande@iu.edu}

% The default list of authors is too long for headers}
\renewcommand{\shortauthors}{S. Khandelwal}

\begin{abstract}
	
In today's world data has different format (structured data, semi-structured and
unstructured data), high volume (data size is upto hundreds of terabytes)
and high velocity (amount of data getting generated per second is too
huge). Also, There is massive amount of on-premise and cloud-based data. 
Every organization has a necessity to integrate such kind of
data between on-premise and cloud-based systems for synchronization, replication or building warehouse for analysis and reporting purpose. Informatica Intelligent Cloud Services (IICS) provide complete suite of integration for such kind of diverse and unique requirement.

\end{abstract}

\keywords{hid-sp18-511, IICS, Informatica, ETL, Data Integration}


\maketitle


\section{Introduction}

Informatica Intelligent Cloud Services is built on micro service
architecture and provide capability of data integration between
cloud-based and on-premise systems. Informatica Intelligent Clod Services IICS~\cite{hid-sp18-511-iics} provide integration capability in four areas mainly Integration Cloud,
Data Quality and Governance Cloud, Master Data Management Cloud, and
Data Security Cloud.

\subsection{Integration Cloud}

\subsection{Data Quality and Governance Cloud}

\subsection{Master Data Management Cloud}

\subsection{Data Security Cloud}

\section{Figures}

In Figure~\ref{f:fly} see the product of Informatica Integration
Cloud Services (IICS)~\cite{hid-sp18-511-iics}.

\begin{acks}

The author would like to thank Dr.~Gregor~von~Laszewski for his
support and suggestions to write this paper.

\end{acks}

\bibliographystyle{ACM-Reference-Format}
\bibliography{report} 
